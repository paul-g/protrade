\documentclass[10pt]{report}
\usepackage{listings}
\usepackage{graphicx}
\usepackage{url}

% global commands

\newcommand{\javalst}[2]{
  \lstset{language=Java,captionpos=b,tabsize=4,frame=single,numbers=left,
    numberstyle=\tiny,numbersep=10pt,breaklines=true,showstringspaces=false,
    basicstyle=\footnotesize,emph={label}, caption={#1}, label={#2}}
}
\newcommand{\initlisting}[2]{
  \lstset{language={#1},captionpos=b,tabsize=4,frame=single,numbers=left,
    numberstyle=\tiny,numbersep=10pt,breaklines=true,showstringspaces=false,
    basicstyle=\footnotesize,emph={label}, caption={#2}}
}

\newcommand{\nm}{{\bf proTrade}}
\newcommand{\nmsp}{{\nm \ }}
% end commands
\setlength{\parskip}{0.3cm}
\setlength{\parindent}{0cm}

\begin{document}

\title{\nmsp - A Professional Tennis Trading Environment}

\author{Corina Ciobanu \and Iskander Orazbekov \and Mir Sahin \and Paul Grigoras \and Radu Baltean-Lugojan}

\date{\today}         % inserts today's date

\maketitle            % generates the title from the data above

\begin{abstract}
Tennis trading is a steadily growing market on the Betfair Exchange, with more than 70\% of bets being placed in-play. In order to maintain market liquidity, exchanges must attract customers for example, by supplying them with better tools. By providing more information and better visualization techniques such a tool can help the trader improve his understanding and predict the market evolution, which should (potentially) lead to an increased profit.

In the case of tennis trading, the information required to understand and predict the evolution of the market associated with a particular match consists of the live score, player statistics, potentially a live video feed and - of course - the market data (evolution of betting odds). Ideally, this information is desired for both historical and live matches.

At the moment, no application provides the entire information. A number of solutions exist which allow visualization of historical market data, but they generally lack the more specific, tennis related data. For example the Fracsoft Data Viewer (\cite{site-fracsoft}) does not correlate market data with match data (scores, player statistics). BetAngel (\cite{site-betangel}) provides some tennis related data and prediction, but relies on the user to input the current match score by pushing buttons. This not entirely suited for the high rate with which data is updated. Ideally all the information should be automatically provided to the user to increase update speed and enhance usability.

\nmsp means to fill this gap, by providing all the information, betting and prediction functionalities for both historical and live matches, in an entirely automated fashion.

\end{abstract}

\tableofcontents

\chapter{Introduction - High Level, Nontechnical Description}

{\bf
\begin{itemize} 
\item Why you should buy this product/listen to this presentation? 
\item What is the functionality of the product?
\end{itemize}
}


\chapter{Technical Description}
{\bf
  \begin{itemize}
    \item introduction into technologies used
    \item Design of your software, possibly including a diagram of the major components of the project 
    \item Main achievements
  \end{itemize} 
}

\chapter{Softare Engineering Issues}
{\bf 
\begin{itemize} 
\item What technology was used and why
\item What other technology was considered but not used and why 
\item Any technical challenges encountered and how addressed 
\item Any risks anticipated, and how mitigated?
\item Any collaboration/coordination difficulties encountered and how addressed 
\item Development and testing methods and/or tools used; 
\item comparison of plans with actual achievements 
\item Estimates of length of code in each of the components, or any other comparable measure of the effort required.
\item Summary of each team member's contributions 
\end{itemize}
}


\chapter{Validation}
{\bf 
\begin{itemize} 
\item How did you validate your product?
\end{itemize} Was the project successful? What did you learn? What might you have done differently? }

\chapter{Conclusions}
{\bf 
  \begin{itemize} 
  \item Was the project successful?
  \item What did you learn?
  \item What might you have done differently?
  \end{itemize}  
}

\begin{thebibliography}{9}
\bibitem{bk-testing}
  Steeve Freeman, Nat Pryce,
  \emph{Growing Object-Oriented Software, Guided by Tests}, Addison-Wesley 2010

\bibitem{web-cyccom}
  Robert Chatley's course on Software Engineering Methods
  \url{http://en.wikipedia.org/wiki/Cyclomatic_complexity},
  Imperial College London

\bibitem{bk-aglsam}
  Jonathan Rasmusson,
  \emph{The Agile Samurai},
  Pragmatic Bookshelf,
  October 2010.

\bibitem{bk-aglflh}
  Jeff Langr and Tim Ottinger,
  \emph{Agile in a Flash},
  Pragmatic Bookshelf, 
  January 2011.

\bibitem{web-rbc}
  Robert Chatley's course on Software Engineering Methods
  \url{http://www.doc.ic.ac.uk/~rbc/302/},
  Imperial College London

\bibitem{site-fracsoft}
  Site of Fracsoft, authors of Fracsoft Data Viewer
  \url{http://www.fracsoft.com}

\bibitem{site-betangel}
  Site of BetAngel
  \url{http://www.betangel.com}
  
\bibitem{wiki-swt}
  SWT - Wikipedia
  \url{http://en.wikipedia.org/wiki/Standard_Widget_Toolkit}

\end{thebibliography}

\clearpage
\appendix

{\bf 
\begin{itemize}
\item Appendix: The appendix is optional, and does not count towards the 45 pages. It may contain thing like: User guide, installation instructions; more extensive design, testing, statistics etc.
\end{itemize}
}

\chapter{Tennis Rules}

\chapter{About Betting Odds and Tennis Trading}

\chapter{Worklogs}

\chapter{}

\end{document}

